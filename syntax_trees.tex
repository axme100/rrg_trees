\documentclass[11pt, spanish]{report}
\usepackage[spanish]{babel}
%Normally I would have used inputenc as overleaf reccomends, but for some reason when using the packages rrg-trees and their dependencies, I coulnd't get the Spanish accent marks to show up
\usepackage[T1]{fontenc}
\selectlanguage{spanish}
\usepackage{rrgtrees}
% These packages are required
\usepackage{pst-node}
\usepackage{pst-tree}
\usepackage{verbatim}

\begin{document}
\section*{Mi abuela conoció a Juan Pérez en su juventud}
\psset{treesep=2ex}
  \SENTENCE{
    \CLAUSE{
      \CORE[COREa]{
        \skiplevels{2}
        \OPR{2}{Mi}
        \ARG{\WORD(NP){{abuela}}}
        \NUC{conoció}
        \OPR{2}{a}
        \ARG[ACD]{\WORD(NP){Juan Pérez}}
        \endskiplevels
      }
    }
  }
\rPERIPH[a]{3}{PP}{
	\CORE{
    \NUC[P]{en}
    \ARG{\WORD(NP){su juventud}}
	}
}
\dolinks
\\
\\
\\
\\
\\
\newpage
\section*{El médico ya vio varios pacientes}
\psset{treesep=2ex}
\SENTENCE{%
  \CLAUSE{%
    \CORE[COREb]{%
      \OPR{2}{El}%
      \ARG{\WORD(NP){médico}}%
      \rPERIPH[b]{1}{\rnode{FA}{FA}}{\WORD{ya}}%
      \NUC{vio}%
      \ARG[ACD]{\WORD(NP){varias pacientes}}%
    }%
  }%
}
\pnode[-4,0](COREb){P0}
\rput(P0){\textsc{Periphery}}
\ncline[nodesepA=1]{->}{P0}{COREb}
\ncangles[armA=0.5,angleA=-90,angleB=90]{P0}{FA}
\\
\\
\\
\\
\\
\newpage
\section*{Los bomberos salieron de la estación}
\psset{treesep=2ex}
  \SENTENCE{
    \CLAUSE{
      \CORE[COREa]{
        \skiplevels{2}
        \OPR{2}{Los}
        \ARG{\WORD(NP){{bomberos}}}
        \NUC{salieron}
        \ARG{\FanEnd{PP}{de la estación}}
        \endskiplevels
      }
    }
  }
\dolinks
\\
\\
\\
\\
\\
\newpage
\section*{No sacrificaré mi vida por Juan}
\psset{treesep=2ex}
  \SENTENCE{
    \CLAUSE{
      \CORE[COREa]{
        \skiplevels{2}
        \ARG{\WORD(NP){{[yo]}}}
        \OPR{2}{no}
        \NUC{sacrificaré}
        \ARG[ACD]{\WORD(NP){mi vida}}
        \endskiplevels
      }
    }
  }
\rPERIPH[a]{3}{PP}{
	\CORE{
    \NUC[P]{por}
    \ARG{\WORD(NP){Juan}}
	}
}
\dolinks
\\
\\
\\
\\
\\
\newpage
\section*{A las niñas nos escondían adentro de la casa}
\psset{treesep=2ex}
  \SENTENCE{
    \CLAUSE{
      \CORE[COREa]{
      	\ARG{\WORD(NP){{*A las niñas}}}
        \ARG{\WORD(NP){{*nos}}}
        \NUC{escondían}
        \ARG[ACD]{\WORD(PP){adentro de la casa}}
      }
    }
  }
\dolinks

\newpage
%Los argumentos marcados con * tienen que representar el mismo referente. Esta oración muestra algún fenómeno que se podría llamar duplicación del sujeto.
\section*{Escoge al gatito que más te guste}
\psset{treesep=2ex}
  \SENTENCE{
    \CLAUSE{
      \CORE[COREa]{
        \ARG{\WORD(NP){{[el que escoge]}}}
        \NUC{escoge}
        \ARG{\FanEnd{NP}{al gatito que más te guste}}
      }
    }
  }
\dolinks
\\
\\
\\
\\
\\
% Juan se olvidó del libro en la mesa
\psset{treesep=2ex}
  \SENTENCE{
    \CLAUSE{
      \CORE[COREa]{
        \ARG{\WORD(NP){{Juan}}}
        \NUC{se olvidó}
        \ARG{\FanEnd{NP}{del libro en la mesa}}
      }
    }
  }
\dolinks
\\
\\
\\
\\
\\
%Con estos lentes puedo ver los bichitos en las plantas
% No sacrificaré mi vida por Juan
\psset{treesep=2ex}
\lPERIPH[a]{3}{PP}{
	\CORE{
    \NUC[P]{Con}
    \ARG{\WORD(NP){estos lentes}}
	}
}
  \SENTENCE{
    \CLAUSE{
      \CORE[COREa]{
        \skiplevels{2}
        \ARG{\WORD(NP){{[yo]}}}
        \OPR{2}{puedo}
        \NUC{ver}
        \ARG{\FanEnd{NP}{los bichitos en las plantas}}
        \endskiplevels
      }
    }
  }
\dolinks
\\
\\
\\
\\
\\
% El año pasado, los biólogos identificaron algunas de las especies
\psset{treesep=2ex}
\lPERIPH[a]{3}{NP}{
	\CORE{
    \NUC[N]{el año pasado}
	}
}
  \SENTENCE{
    \CLAUSE{
      \CORE[COREa]{
        \skiplevels{2}
        \ARG{\WORD(NP){{los biólogos}}}
        \NUC{identificaron}
        \ARG{\FanEnd{NP}{algunas de las especies}}
        \endskiplevels
      }
    }
  }
\dolinks
\\
\\
\\
\\
\\
% Saluda a la doctora a la entrada
\psset{treesep=2ex}
  \SENTENCE{
    \CLAUSE{
      \CORE[COREa]{
        \skiplevels{2}
        \ARG{\WORD(NP){{[El que saluda]}}}
        \NUC{saluda}
        \OPR{2}{a}
        \ARG{\WORD(NP){{la doctora}}}
        \endskiplevels
      }
    }
  }
\rPERIPH[a]{3}{PP}{
	\CORE{
    \NUC[P]{a}
    \ARG{\WORD(NP){{la entrada}}}
	}
}  
\dolinks
\\
\\
\\
\\
\\
% Nadie confía en ti desde el accidente
\psset{treesep=2ex}
  \SENTENCE{
    \CLAUSE{
      \CORE[COREa]{
        \skiplevels{2}
        \ARG{\WORD(NP){{Nadie}}}
        \NUC{confia}
        \ARG{\FanEnd{PP}{en ti   }}
        \endskiplevels
      }
    }
  }
\rPERIPH[a]{3}{PP}{
	\CORE{
    \NUC[P]{desde}
    \ARG{\WORD(NP){{el accidente}}}
	}
}  
\dolinks
\\
\\
\\
\\
\\
% Le acaricié la cabeza a Juan
\psset{treesep=2ex}
  \SENTENCE{
    \CLAUSE{
      \CORE[COREa]{
        \skiplevels{2}
        \ARG{\WORD(NP){{[yo]}}}
        \ARG{\WORD(NP){{*le}}}
        \NUC{acaricié}
        \ARG{\FanEnd{NP}{la cabeza}}
        \ARG{\WORD(NP){{*a Juan}}}
        \endskiplevels
      }
    }
  }
\dolinks
\\
\\
\\
\\
\\
% Tú tienes miedo de la obscuridad
\psset{treesep=2ex}
  \SENTENCE{
    \CLAUSE{
      \CORE[COREa]{
      	\skiplevels{2}
        \ARG{\WORD(NP){{Tú}}}
        \NUC[semi-copulativo]{tienes miedo}
        \endskiplevels
      }
    }
  }
\rPERIPH[a]{3}{PP}{
	\CORE{
    \NUC[P]{de}
    \ARG{\WORD(NP){la oscuridad}}
	}
}
\dolinks
\\
\\
\\
\\
\\
% En las mañanas saluda la doctora
% El año pasado, los biólogos identificaron algunas de las especies
\psset{treesep=2ex}
\lPERIPH[a]{3}{PP}{
	\CORE{
    \NUC[P]{En}
    \ARG{\FanEnd{NP}{las mañanas}}
	}
}
  \SENTENCE{
    \CLAUSE{
      \CORE[COREa]{
        \skiplevels{2}
        \NUC{ saluda}
        \ARG{\FanEnd{NP}{la doctora}}
        \endskiplevels
      }
    }
  }
\dolinks
\\
\\
\\
\\
\\
% Terminé con mi novio
\psset{treesep=2ex}
  \SENTENCE{
    \CLAUSE{
      \CORE[COREa]{
        \ARG{\WORD(NP){{[Yo]}}}
        \NUC{terminé}
  		\ARG{\FanEnd{PP}{con mi novio}}
      }
    }
  }
\dolinks
\\
\\
\\
\\
\\
% Me saqué el sueter con dificultad 
\psset{treesep=2ex}
  \SENTENCE{
    \CLAUSE{
      \CORE[COREa]{
      	\skiplevels{2}
        \ARG{\WORD(NP){{[Yo]}}}
        \NUC{me saqué}
        \ARG{\FanEnd{NP}{el sueter}}
        \endskiplevels
      }
    }
  }
\rPERIPH[a]{3}{PP}{
	\CORE{
    \NUC[P]{con}
    \ARG{\WORD(NP){dificultad}}
	}
}
\dolinks
\\
\\
\\
\\
\\
% Le estoy tejiendo una bufanda
\psset{treesep=2ex}
  \SENTENCE{
    \CLAUSE{
      \CORE[COREa]{
      	\skiplevels{2}
        \ARG{\WORD(NP){{[Yo]}}}
        \ARG{\WORD(NP){{le}}}
        \NUC{estoy tejiendo}
        \ARG{\FanEnd{NP}{una bufanda}}
        \endskiplevels
      }
    }
  }
\dolinks
\\
\\
\\
\\
\\
% Silvia te mandó un abrazo conmigo
\psset{treesep=2ex}
  \SENTENCE{
    \CLAUSE{
      \CORE[COREa]{
      	\skiplevels{2}
        \ARG{\WORD(NP){{Silvia}}}
        \ARG{\WORD(NP){{te}}}
        \NUC{mandó}
        \ARG{\FanEnd{NP}{un abrazo}}
        \endskiplevels
      }
    }
  }
\rPERIPH[a]{3}{NP}{
	\CORE{
    \NUC[P]{conmigo}
	}
}
\dolinks
\\
\\
\\
\\
\\
% María vino de su casa al trabajo
\psset{treesep=2ex}
  \SENTENCE{
    \CLAUSE{
      \CORE[COREa]{
      	\skiplevels{2}
        \ARG{\WORD(NP){{María}}}
        \NUC{vino}
        \endskiplevels
      }
    }
  }
\rPERIPH[a]{3}{PP}{
	\CORE{
    \NUC[P]{de su casa}
	}
}
\dolinks
\\
\\
\\
\\
\\
% Sofía depende de las semillas de chia para vivir
\psset{treesep=2ex}
  \SENTENCE{
    \CLAUSE{
      \CORE[COREa]{
      	\skiplevels{2}
        \ARG{\WORD(NP){{Sofía}}}
        \NUC{depende}
        \ARG{\FanEnd{PP}{de las semillas de chia}}
        \endskiplevels
      }
    }
  }
\rPERIPH[a]{3}{PP}{
	\CORE{
    \NUC[P]{   para vivir}
	}
}
\dolinks
\end{document}